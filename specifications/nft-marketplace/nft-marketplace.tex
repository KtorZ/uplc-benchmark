\let\mypdfximage\pdfximage
\def\pdfximage{\immediate\mypdfximage}

\documentclass{article}

\usepackage{lipsum}
\usepackage{amsmath}
\usepackage{amssymb}
\usepackage{amsthm}
\usepackage{tikz}
\usepackage{xcolor}
\usepackage{float}
\usepackage{array}
\usepackage{tabu}
\usepackage{amsfonts}
\usepackage{bold-extra}
\usepackage{multirow}
\usepackage[colorlinks=true,linkcolor=webgreen,filecolor=webbrown,citecolor=webgreen]{hyperref}
\usepackage[capitalise,noabbrev]{cleveref}
\usepackage{longtable}
\usepackage{listings}
\usepackage[a4paper]{geometry}

\tabulinesep=1.2mm

\title{NFT Marketplace Specification}
\author{Tomasz Maciosowski}
\date{}

\begin{document}

\begin{center}
  \vskip 10mm {\LARGE\bf
      NFT Marketplace Specification
    }
  \vskip 10mm
  Tomasz Maciosowski \\
  MLabs\\
  \href{mailto:tomasz@mlabs.city}{\tt tomasz@mlabs.city} \\
\end{center}

\vskip 5mm

\begin{abstract}
  This document describes a simple NFT marketplace escrow validator designed for the Cardano blockchain.
  This validator allows users to create and cancel sale orders for NFTs at a predetermined fixed price in any Cardano native tokens.
  The implementation of this validator will be carried out across multiple programming languages, compiled into UPLC, and tested with a uniform test suite.
  While the validator is engineered with security in mind, certain functionalities necessary for a production environment may be absent.
\end{abstract}

\section{Introduction}

The validator allows users to lock a certain token in the script-owned UTxO, with datum specifying the NFT to be sold, and a price in any Cardano native token, address where the seller will receive the payment, and a key that can cancel the sale order and return the locked token to the seller.

\section{Validator}

The validator is not parameterized. Despite validator being intended as NFT marketplace, it operates on the UTxO level thus can be used to sell any bundle of tokens.

\subsection{Data Types}
\begin{lstlisting}[language=Haskell]
data Redeemer = Buy | Cancel

data Datum = Datum
  { price :: Value
  , seller :: Address
  , cancelKey :: PubKeyHash
  }
\end{lstlisting}

\subsection{Validation Checks}

For a transaction to be valid, the following checks must pass:

\textbf{Case 1: Cancel}
\begin{itemize}
  \item Transaction is signed by \verb|cancelKey|.
\end{itemize}


\textbf{Case 2: Buy}
\begin{itemize}
  \item \verb|price| is paid to \verb|seller| with a UTxO reference in inline datum.
        We are attaching a unique datum to prevent double satisfaction attack.
\end{itemize}



\end{document}